% 导言区
\documentclass{article}

\usepackage{ctex}
\usepackage{amsmath}

%标题、作者及日期
\title{数学模式中的矩阵}
\author{张三丰}
\date{\today}


% 正文区(文稿区)
\begin{document}
	\maketitle
	%矩阵环境中,&分隔列 \\分隔行
	\[
	\begin{matrix}  
	0 & 1 \\ 
	1 & 0  
	\end{matrix}  
	\] \quad
	
	\[
	\begin{pmatrix}
	0 & 1 \\
	1 & 0
	\end{pmatrix}
	\] \quad
	
	\[
	\begin{bmatrix}
	0 & 1 \\
	1 & 0
	\end{bmatrix}
	\] \quad
	
	\[
	\begin{Bmatrix}
	0 & 1 \\
	1 & 0
	\end{Bmatrix}
	\] \quad
	
	\[
	\begin{vmatrix}
		0 & 1 \\
		1 & 0
	\end{vmatrix}
	\] \quad
	
	\[
	\begin{Vmatrix}
	0 & 1 \\
	1 & 0
	\end{Vmatrix}
	\] \quad
	%常用省略号:\dots \vdots $\ddots
	\[
	A = \begin{bmatrix}
	a_{11} & \dots & a_{1n} \\
	& \ddots & \vdots \\
	0 & \dots & a_{nn}
	%在数学模式中可以用times命令排版乘号
	\end{bmatrix}_{n \times n}
	\] \quad
	
	%分块矩阵(矩阵嵌套)
	\[
	\begin{pmatrix}
	\begin{matrix} 1&0 \\0&1 \end{matrix}
	& \text{\Large 0} \\
	\text{\Large 0} & \begin{matrix}
	1&0\\0&-1 \end{matrix}
	\end{pmatrix}
	\]
	
\end{document}