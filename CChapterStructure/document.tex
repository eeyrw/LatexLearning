% 导言区
\documentclass[12pt]{article}%{ctexart}

\usepackage{ctex}%引入中文包,使得中文可以正常显示

%标题、作者及日期
\title{标题及目录}
\author{张三丰}
\date{\today}

% 正文区(文稿区)
\begin{document}
	\maketitle
	%生成文档目录
	\tableofcontents
	%构建各章节的一级小结
	\section{摘要}
	%换行符号:空行(多个空行等同一个空行)或者\\
	车牌识别系统(License Plate Recognition 简称LPR)技术基于数字图像处理,是智能交通系统中的关键技术,同时他的发展也十分迅速,已经逐渐融入到我们的现实生活中。文章介绍了车牌识别系统的意义、图像去噪处理以及图像二值化方法,并通过仿真试验模拟了图像处理的过程。本文所做的工作在于前期的图像预处理工作。本次设计着重在于图像识别方面, 中心工作都为此而展开,文中没有进行车牌的定位处理,而是采用数码相机直接对牌照进行正面拍照,获取原始车牌图像。之后利用Matlab编程对图片进行了大小的调整、彩色图片转化成灰度图片、图片去噪、以及图片二值化等工作。其中,去噪与二值化是关系图像识别率的关键。
	
	%通常为了保证编写的结构清晰,我们使用空行进行分隔
	关键字:车牌识别系统;图像预处理;字符识别;Matlab;去噪;二值化
	\section{引言}
	\section{实验方法}
		%构建二级小节
		\subsection{数据}
		\subsection{图表}
			%构建三级小节
			\subsubsection{实验条件}
			\subsubsection{实验过程}
	\section{实验结果}
	\section{结论}
	\section{致谢}
\end{document}