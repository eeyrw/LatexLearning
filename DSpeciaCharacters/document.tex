% 导言区
\documentclass[12pt]{article}

\usepackage{ctex}

%标题、作者及日期
\title{特殊字符}
\author{张三丰}
\date{\today}

% 正文区(文稿区)
\begin{document}
	\maketitle
	\section{空白符号}
	%1em(代表当前字体中M的宽度)
	ab		%没有空格
	
	a\,b or a\thinspace b	%中等空格,1/6m宽度
	
	a\ b	%1/3m宽度
	
	a~b	%硬空格
	
	a\enspace b	%1/2m宽度
	
	a \quad b	%quad空格,一个m的宽度
	
	a \qquad b	%两个quad空格,两个m的宽度	
	
	%1pc = 12pt = 4.218mm
	a\kern 1pc b
	
	a\kern -1em b
	
	a\hskip 1em b
	
	a\hspace{35pt}b
	
	%占位宽度
	a\hphantom{xyz}b
	
	%弹性长度,用于撑满整个空间
	a\hfill b
	
	Using the recently proposed non-local (NL) model [43],we achieve 40.3 mask AP and 45.0 box AP. This result is without test-tim augmentation, and the method runs at 3fps on an Nvidia Tesla P100 GPU at test time.
	
	智能交通系统(ITS)是当今世界交通管理体系发展的必然趋势,而作为智能交通系统中的重要组成部分之一的车牌自动识别技术,目前已被广泛应用于城市道路监控、高速公路收费与监控、小区与停车场出入口管理、公安治安卡口等场合,成为研究的热点。
	
	智能交通系统(ITS)是当今世界交通管理体系发展的必然趋势。This result is without test-time augmentation, and the method runs at 3fps on an Nvidia Tesla P100 GPU at test time.高速公路收费与监控、小区与停车场出入口管理、公安治安卡口等场合,成为研究的热点。
	\section{\LaTeX 控制符}
	\# \quad \$ \quad \% \quad \{ \} \quad \~{} \quad \_{} \quad \^{} \quad \textbackslash \quad \& 
	
	\section{排版符号}
	\S \quad \P \quad \dag \quad \ddag \quad \copyright \quad \pounds
	
	\section{\TeX 标志符号}
	\TeX{} \quad \LaTeX{} \quad \LaTeXe{}
	
	\section{引号}
	%ESC 按键下面的一个为左单引号,两个为左双引号  通常用的引号一个为右单引号,两个为右双引号,所以在LaTeX当中引号是严格分左右的
	` ' `` '' ``你好''

	\section{连字符}
	- \quad -- \quad ---
	
	\section{非英文字符}
	\oe \quad \OE \quad \ae \quad \AE \quad \aa \quad \AA \quad \o \quad \O \quad \l \quad \L \quad \ss \quad \SS 
	
	\section{重音符号,以o为例}
	\`o \quad \'o \quad \^o \quad \~o \quad \=o \quad \.o \quad \u{o} \quad \v{o} \quad \H{o} \quad \r{o} \quad \t{o} \quad \b{o} \quad \c{o} \quad \d{o} \quad
\end{document}